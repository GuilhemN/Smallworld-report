
\chapter{Help Toolkit (by Paulette Vazquez \& Guilhem Niot)}


\section{First configuration of the Raspberry Pi}

In order to set up the Raspberry Pi, you will need to obtain a shell on it. You can either connect it to a screen, or activate SSH using a SD card reader and configure the network using \ref{conf:network_template}. \\
If you choose to connect it to a screen, you can enable SSH using the command \emph{raspi-config} in a terminal, this will allow you to manage the Raspberry remotely when connected to the same network. \\

You will also require to configure internet for the set up, in order to retrieve the libraries and dependencies. A simple way of doing that is to set up a hotspot on your phone, and to configure it on the Raspberry Pi using \ref{conf:network_template}.

